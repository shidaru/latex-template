% -*- coding: utf-8 -*-
\documentclass[uplatex]{jsarticle}
\和暦
\usepackage[dvipdfmx]{graphicx}
\usepackage{graphicx}
\usepackage[noalphabet]{pxchfon}
\setminchofont{ipam.ttf}
\setgothicfont{ipag.ttf}
\usepackage{amssymb, amsmath}
\usepackage[at]{easylist}


\begin{document}
\title{\Huge \LaTeX 記述例}
\author{shidaru}
\date{\today}
\maketitle

\section{はじめに}
\LaTeX の例(日本語用)

\subsection{箇条書き}
\begin{itemize}
  \item item1
  \item item2
  \item ...
  \item itemN
\end{itemize}

\begin{easylist}
 @ 北日本
 @@ 北海道
 @@ 東北
 @@@ 宮城
 @@@ 山形
 @ 東日本
\end{easylist}

\subsection{数式}
\begin{equation}
  x^2 - 6x + 1 = 0
\end{equation}

\begin{eqnarray}
  2x_1 + x_2 & = & 5 \\
  2x_2 & = & 2
\end{eqnarray}

\begin{eqnarray}
  2x_1 + x_2 & = & 5 \nonumber \\
  2x_2 & = & 2
\end{eqnarray}

\subsection{表}
\begin{table}[htp]
  \caption{C言語の代表的なデータの型}
  \label{table:data_type}
  \centering
  \begin{tabular}{lcr}
    \hline
    データの型  & 宣言  &  ビット幅  \\
    \hline \hline
    文字型  & char  & 8 \\
    整数型  & int   & 32 \\
    倍精度実数型  & double  & 64 \\
    倍々精度実数型  &  long double  &  96 \\
    \hline
  \end{tabular}
\end{table}

\subsection{画像}
\begin{figure}[htp]
\begin{center}
 \includegraphics[width=7cm]{imgs/latex.jpg}
\end{center}
 \caption{\LaTeX}
 \label{fig:nof}
\end{figure}


\section{おわりに}
参考になる:\TeX wiki\cite{tw}。

% 参考文献
% junsrt: 引用した順番に記載
% reference: hoge.bibのhogeにあたる
\bibliography{reference}
\bibliographystyle{junsrt}

\end{document}
